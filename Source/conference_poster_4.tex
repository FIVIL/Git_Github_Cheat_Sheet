%%%%%%%%%%%%%%%%%%%%%%%%%%%%%%%%%%%%%%%%%%%%%%%%
% Python Cheat Sheet
% baposter Landscape Poster
% LaTeX Template
% Version 1.0 (11/06/13)
% baposter Class Created by:
% Brian Amberg (baposter@brian-amberg.de)
% This template has been downloaded from:
% http://www.LaTeXTemplates.com
% License:
% CC BY-NC-SA 3.0 (http://creativecommons.org/licenses/by-nc-sa/3.0/)
% Edited by Michelle Cristina de Sousa Baltazar
%%%%%%%%%%%%%%%%%%%%%%%%%%%%%%%%%%%%%%%%%%%%%%%%

%----------------------------------------------------------------
%	PACKAGES AND OTHER DOCUMENT CONFIGURATIONS
%----------------------------------------------------------------

\documentclass[landscape,a0paper,fontscale=0.285]{baposter} % Adjust the font scale/size here
\title{Python Cheat Sheet New}
\usepackage[brazilian]{babel}
%
\usepackage[utf8]{inputenc}

\usepackage{graphicx} % Required for including images
\graphicspath{{figures/}} % Directory in which figures are stored

\usepackage{xcolor}
\usepackage{colortbl}
\usepackage{tabu}

\usepackage{mathtools}
%\usepackage{amsmath} % For typesetting math
\usepackage{amssymb} % Adds new symbols to be used in math mode

\usepackage{booktabs} % Top and bottom rules for tables
\usepackage{enumitem} % Used to reduce itemize/enumerate spacing
\usepackage{palatino} % Use the Palatino font
\usepackage[font=small,labelfont=bf]{caption} % Required for specifying captions to tables and figures

\usepackage{multicol} % Required for multiple columns

\usepackage{hyperref}
\setlength{\columnsep}{1.5em} % Slightly increase the space between columns
\setlength{\columnseprule}{0mm} % No horizontal rule between columns

\usepackage{tikz} % Required for flow chart
\usetikzlibrary{decorations.pathmorphing}
\usetikzlibrary{shapes,arrows} % Tikz libraries required for the flow chart in the template

\newcommand{\compresslist}{ % Define a command to reduce spacing within itemize/enumerate environments, this is used right after \begin{itemize} or \begin{enumerate}
\setlength{\itemsep}{1pt}
\setlength{\parskip}{0pt}
\setlength{\parsep}{0pt}
}

% \definecolor{lightblue}{rgb}{0.145,0.6666,1} % Defines the color used for content box headers
\definecolor{almond}{rgb}{0.94, 0.87, 0.8}
\definecolor{amethyst}{rgb}{0.6, 0.4, 0.8}
\definecolor{antiquefuchsia}{rgb}{0.57, 0.36, 0.51}
\definecolor{applegreen}{rgb}{0.55, 0.71, 0.0}
\definecolor{asparagus}{rgb}{0.53, 0.66, 0.42}
\definecolor{babyblueeyes}{rgb}{0.63, 0.79, 0.95}
\definecolor{babypink}{rgb}{0.96, 0.76, 0.76}
\definecolor{bananamania}{rgb}{0.98, 0.91, 0.71}
\definecolor{blanchedalmond}{rgb}{1.0, 0.92, 0.8}
\definecolor{burlywood}{rgb}{0.87, 0.72, 0.53}
\definecolor{cambridgeblue}{rgb}{0.64, 0.76, 0.68}
\definecolor{champagne}{rgb}{0.97, 0.91, 0.81}
\definecolor{darkkhaki}{rgb}{0.74, 0.72, 0.42}
\definecolor{darkseagreen}{rgb}{0.56, 0.74, 0.56}
\definecolor{floralwhite}{rgb}{1.0, 0.98, 0.94}
\definecolor{antiquewhite}{rgb}{0.98, 0.92, 0.84}
\definecolor{aurometalsaurus}{rgb}{0.43, 0.5, 0.5}

\begin{document}
{\fontfamily{helvet}\selectfont

\begin{poster}
{
headerborder=closed, % Adds a border around the header of content boxes
colspacing=0.8em, % Column spacing
bgColorOne=almond, % Background color for the gradient on the left side of the poster
bgColorTwo=aurometalsaurus, % Background color for the gradient on the right side of the poster
borderColor=babypink, % Border color
headerColorOne=darkseagreen, % Background color for the header in the content boxes (left side)
headerColorTwo=babypink, % Background color for the header in the content boxes (right side)
headerFontColor=white, % Text color for the header text in the content boxes
boxColorOne=white, % Background color of the content boxes
textborder=roundedleft, % Format of the border around content boxes, can be: none, bars, coils, triangles, rectangle, rounded, roundedsmall, roundedright or faded
eyecatcher=true, % Set to false for ignoring the left logo in the title and move the title left
headerheight=0.1\textheight, % Height of the header
headershape=roundedright, % Specify the rounded corner in the content box headers, can be: rectangle, small-rounded, roundedright, roundedleft or rounded
headerfont=\Large\bf\textsc, % Large, bold and sans serif font in the headers of content boxes
%textfont={\setlength{\parindent}{1.5em}}, % Uncomment for paragraph indentation
linewidth=2pt % Width of the border lines around content boxes
}
%----------------------------------------------------------------
%	TÍTULO
%----------------------------------------------------------------
{\bf\textsc{Git and Github Cheat Sheet}\vspace{0.5em}} % Poster title
{\textsc{\{ Git \ \ \ \ \ \& \ \ \ \ \   Github \ \ \ \ \ Cheat \ \ \ \ \ Sheet\} \hspace{12pt}}}
{\textsc{Sepideh K. Gharamaleki  \hspace{12pt}}} 


%------------------------------------------------
% Git Config
%------------------------------------------------
\headerbox{Config, Create and Branch}{name=objectives,column=0,row=0}{

%--------------------------------------
\colorbox[HTML]{A3C1AD}{\makebox[\textwidth-2\fboxsep][l]{\bf - Git Config}}
\begin{itemize}\compresslist
\item git config ${--}$global user.email “youremail”
\item Make vscode default instead of VIM:\\
git config ${--}$global core.editor "code ${--}$--wait"

\end{itemize}


%--------------------------------------
%--------------------------------------
\colorbox[HTML]{A3C1AD}{\makebox[\textwidth-2\fboxsep][l]{\bf - SSH Config}}
\begin{itemize}
\item Click \href{https://docs.github.com/en/authentication/connecting-to-github-with-ssh/checking-for-existing-ssh-keys}{here.}
\end{itemize}

%--------------------------------------
\colorbox[HTML]{A3C1AD}{\makebox[\textwidth-2\fboxsep][l]{\bf - Create Repo:}}
\begin{itemize}\compresslist
\item git status
\item git init 


\end{itemize}

%--------------------------------------

%--------------------------------------
\colorbox[HTML]{A3C1AD}{\makebox[\textwidth-2\fboxsep][l]{\bf - Commit:}}
\begin{itemize}\compresslist
\item git add file

\item git commit -m “message”/ git commit 



\end{itemize}

%--------------------------------------

%--------------------------------------
\colorbox[HTML]{A3C1AD}{\makebox[\textwidth-2\fboxsep][l]{\bf - Log:}}
\begin{itemize}\compresslist
\item git log ${--}$oneline


\end{itemize}

%--------------------------------------

%--------------------------------------
\colorbox[HTML]{A3C1AD}{\makebox[\textwidth-2\fboxsep][l]{\bf - Fixing Commits (just one commit ago)}}
\begin{itemize}\compresslist
\item git add forgotten\_file
\item git commit –amend



\end{itemize}

%-------------------------------------
%--------------------------------------
\colorbox[HTML]{A3C1AD}{\makebox[\textwidth-2\fboxsep][l]{\bf - Create and Switch to Branch}}
\begin{itemize}\compresslist
\item git checkout -b branchname


\item git switch -c branchname

\item If switch before committing changes: doesn’t work if the file is shared between the branches. Refer to stashing changes. 

\end{itemize}

%-------------------------------------

%--------------------------------------
\colorbox[HTML]{A3C1AD}{\makebox[\textwidth-2\fboxsep][l]{\bf - Manipulating Branches}}
\begin{itemize}\compresslist
\item Delete:
git branch -d or -D if branch is not commited
\item Rename:
Switch to branch first: git branch -m newbranchname
\item Merge branches:
We merge to HEAD.
git merge branchname
\item 
git branch -v : shows branchnames hash of commit last commit
\item If conflicts: resolve then add and commit the changes.


\end{itemize}

% %--------------------------------------
% \colorbox[HTML]{CCFFFF}{\makebox[\textwidth-2\fboxsep][l]{\bf - Entrada de Dados:}}
% %\begin{tabular}{lp{2.0cm}lp{3.0cm}|}

% \begin{tabular}{lp{5.3cm}lp{3.0cm}|}
% A = input() & Aguarda a entrada de caracteres armazenados em A \\
% \end{tabular}
% \begin{tabular}{lp{4.7cm}lp{3.0cm}|}
% B = int(input()) & Aguarda a entrada de inteiros armazenados em B \\
% \end{tabular}
% \begin{tabular}{lp{2.6cm}lp{3.0cm}|}
% A,B = map(int,input().split()) & Aguarda a entrada de inteiros  separados por espaço, armazenados em A e B respectivamente \\
% \end{tabular}
% \begin{tabular}{lp{3.0cm}lp{3.0cm}|}
% input("Pressione ENTER") & Aguarda pressionar ENTER para prosseguir - como não declarou nenhuma variável, não irá gravar nada. \\
% \end{tabular}

\vspace{0.0em} % When there are two boxes, some whitespace may need to be added if the one on the right has more content
}

%------------------------------------------------
% Lógica Básica do Python
%------------------------------------------------

\headerbox{Monitor changes}{name=montiorchanges,column=1,row=0,span=2}{

%--------------------------------------
\colorbox[HTML]{A3C1AD}{\makebox[\textwidth-2\fboxsep][l]{\bf - Manipulating Commits}}
\begin{itemize}\compresslist
\item Comparing changes:\\
git diff (commits or files or …)
\item Stashing:
When you have uncommited changes on one branch and want to jump to another:\\git stash
\item Getting back in time:\\
git checkout hashofanoldcommit/HEAD$\sim$1..n\\
- Results in detached head
To get rid of everything after the last commit:
git checkout HEAD 
\item To get rid of everything after any commit:
git reset/revert hashofcommit
\item Revert keeps the commits that are deleted cause it makes a new commit.



\end{itemize}


\colorbox[HTML]{A3C1AD}{\makebox[\textwidth-2\fboxsep][l]{\bf - Rebase}}
\begin{itemize}\compresslist
\item  To merge commits and delete the history of the merged commits:
git rebase master
\item Merge conflicts are resolved with instructions git gives you.

\item Rebase to rewrite commits:\\
git rebase -i HEAD~4
\end{itemize}
}
% %--------------------------------------
\headerbox{Github}{name=github,column=1,row=0.42, span=2, bottomaligned=objectives}{

%--------------------------------------
\begin{itemize}\compresslist
\item 
git remote add origin url
\end{itemize}

\colorbox[HTML]{A3C1AD}{\makebox[\textwidth-2\fboxsep][l]{\bf - Push}}
\begin{itemize}\compresslist
\item 
git push origin <branch>

\end{itemize}

\colorbox[HTML]{A3C1AD}{\makebox[\textwidth-2\fboxsep][l]{\bf - Fetch}}
\begin{itemize}\compresslist
\item Accessing changes on repo without having to commit on local repo. => Fetch:\\
git fetch origin branch
\item to see the changes:\\ git checkout origin/branchname
\item git pull origin branch= git fetch+ git merge
\item wherever you run git pull, the branch will merge there.
Before pushing, pull to see if anyone made changes


\item To resolve conflicts when you pull request:\\
\hspace*{0.15cm} - git fetch origin\\
\hspace*{0.15cm} - git switch my-new-feature\\
\hspace*{0.15cm} - git merge master\\
\hspace*{0.15cm} - fix conflicts!\\
\hspace*{0.15cm} - git switch master\\
\hspace*{0.15cm} - git merge my-new-feature\\
\hspace*{0.15cm} - git push origin master
\end{itemize}
\vspace{0.0em} 



}

% %--------------------------------------
\headerbox{Misc.}{name=misc,column=3,row=0}{

%--------------------------------------
\colorbox[HTML]{A3C1AD}{\makebox[\textwidth-2\fboxsep][l]{\bf - Tagging}}
\begin{itemize}\compresslist
\item 
List all tags: git tag
\item Switch to a tag: git checkout tagname
\item Difference bet. Tags: git diff tagname1 tagname2
\item Create tag: git tag tagname/ git tag -a tagname
\item Earlier commits: git tag tagname commithash
\item You have to push tags: git push ${--}$tags

\end{itemize}

\colorbox[HTML]{A3C1AD}{\makebox[\textwidth-2\fboxsep][l]{\bf - Reflogging}}
\begin{itemize}\compresslist
\item displays a log of all the local reference updates made in your repo.
\item git reflog show Head
\item Reflogs only local logs and expire after 90 days.
\item To retrieve previous commits you can use reflog:git reset –hard master@\{1\}
\item You can undo rebase with reflog.	


\end{itemize}

\colorbox[HTML]{A3C1AD}{\makebox[\textwidth-2\fboxsep][l]{\bf - Aliasing}}
\begin{itemize}\compresslist
\item Alter ~/.gitconfig file. 
\item Can be done on terminal too. 
\item git config ${--}$global alias.nameofalias nameof function


\end{itemize}


}


% %------------------------------------------------
% % Listas no Python
% %------------------------------------------------

% \headerbox{Listas no Python}{name=results,column=2,span=2,row=0}{

% \colorbox[HTML]{CCFFFF}{\makebox[\textwidth-2\fboxsep][l]{\bf - Listas no Python}}
% \linebreak \\
% Listas são compostas por elementos de qualquer tipo (podem ser alteradas) \linebreak \\
% \begin{tabular}{@{}ll@{}}
% \textbf{Manipulação de Listas no Python}\\
% \multicolumn{2}{l}{\cellcolor[HTML]{DDFFFF}Criação} \\
% uma\_lista = [5,3,'p',9,'e'] & cria: [5,3,'p',9,'e'] \\
% \multicolumn{2}{l}{\cellcolor[HTML]{DDFFFF}Acessando} \\
% uma\_lista[0] & retorna: 5 \\
% \multicolumn{2}{l}{\cellcolor[HTML]{DDFFFF}Fatiando} \\
% uma\_lista[1:3] & retorna: [3,'p'] \\
% \multicolumn{2}{l}{\cellcolor[HTML]{DDFFFF}Comprimento} \\
% len(uma\_lista) & retorna: 5 \\
% \multicolumn{2}{l}{\cellcolor[HTML]{DDFFFF}count( item)} \\
% \multicolumn{2}{l}{Retorna quantas vezes o item foi encontrado na lista.} \\
% cont(uma\_lista('p') & retorna: 1 \\
% \multicolumn{2}{l}{Pode ser usado juntamente com a função while para 'andar' pelo comprimento da lista:} \\
% while x < len(uma\_lista): & retorna: [3,'p']\\
% \multicolumn{2}{l}{\cellcolor[HTML]{DDFFFF}Ordenar - sort()} \\
% uma\_lista.sort() & retorna: [3,5,9,'e','p'] \\
% \multicolumn{2}{l}{Ordenar sem alterar a lista} \\
% print(sorted(uma\_lista)) & retorna: [3,5,9,'e','p'] \\
% \multicolumn{2}{l}{\cellcolor[HTML]{DDFFFF}Adicionar - append(item)} \\
% uma\_lista.append(37) & retorna: [5,3,'p',9,'e',37] \\
% \multicolumn{2}{l}{\cellcolor[HTML]{DDFFFF}Inserir - insert(position,item)} \\
% insert(uma\_lista.append(3),200) & retorna: [5,3,200,'p',9,'e'] \\
% \multicolumn{2}{l}{\cellcolor[HTML]{DDFFFF}Retornar e remover - pop(position)} \\
% uma\_lista.pop() & retorna: 'e' e a lista fica [5,3,'p',9] - remove o último elemento \\
% uma\_lista.pop(1) & retorna: 3 e a lista fica [5,'p',9,'e'] - remove o elemento 1 \\
% \multicolumn{2}{l}{\cellcolor[HTML]{DDFFFF}Remover - remove(item)} \\
% uma\_lista.remove('p') & retorna: [5,3,9,'e'] \\
% \multicolumn{2}{l}{\cellcolor[HTML]{DDFFFF}Inserir} \\
% uma\_lista.insert(2,'z') & retorna: [5,'z',3,'p',9,'e'] - insere na posição numerada \\
% \multicolumn{2}{l}{\cellcolor[HTML]{DDFFFF}Inverter - reverse()} \\
% reverse(uma\_lista) & retorna: ['e',9,'p',3,5] \\
% \multicolumn{2}{l}{\cellcolor[HTML]{DDFFFF}Concatenar} \\
% uma\_lista+[0] & retorna: [5,3,'p',9,'e',0] \\
% uma\_lista+uma\_lista & retorna: [5,3,'p',9,'e',5,3,'p',9,'e'] \\
% \multicolumn{2}{l}{\cellcolor[HTML]{DDFFFF}Encontrar} \\
% 9 in uma\_lista & retorna: True \\
% for x in uma\_lista & retorna toda a lista, um elemento por linha \\
% ......print(x) &  
% \end{tabular}
% %------------------------------------------------
% }
\end{poster}


%-----------------------------------
}
%----------------------------------------------------------------
%	REFERENCES  {name=objectives,column=0,row=0}
%----------------------------------------------------------------
%\headerbox{bb}{name=references,column=1,row=0}{}
%----------------------------------------------------------------
%	FUTURE RESEARCH
%----------------------------------------------------------------
%\headerbox{aa}{name=futureresearch,column=1,row=0}{}
%----------------------------------------------------------------
%	CONTACT INFORMATION
%----------------------------------------------------------------
%\headerbox{Contact Information}{name=contact,column=2,span=2,row=0}{}
%----------------------------------------------------------------


\end{document}